\documentclass{article}

% Language setting
% Replace `english' with e.g. `spanish' to change the document language
\usepackage[english]{babel}

% Set page size and margins
% Replace `letterpaper' with `a4paper' for UK/EU standard size
\usepackage[letterpaper,top=1cm,bottom=2cm,left=3cm,right=3cm,marginparwidth=1.75cm]{geometry}

% Useful packages
\usepackage{amsmath}
\usepackage{amssymb}
\usepackage{graphicx}
\usepackage[colorlinks=true, allcolors=blue]{hyperref}

\author{\vspace{-10ex}}
\title{Miscellaneous Problems}
\date{\vspace{-10ex}}

\begin{document}
\maketitle

\section{}
$x,y,t$ are positive real numbers where $t<1$ and they satisfy:
$$\frac{x}{t} + \frac{y}{\sqrt{1-t^2}} = 1$$
Prove that:
$$x^{2/3} + y^{2/3} \leq 1$$

\section{}
Let $(a_n)_{n\geq0}$ be a sequence of real numbers such that for all $n\geq0$
$$a_{n+1} = a_n^2 - 2$$
Find all possible values of $a_0$ for which the sequence will become periodic (i.e. $\exists p,q \in \mathbb{Z}$ such that $p\geq 0, q>0$ and $a_n = a_{n+q} \forall n\geq p$)

\section{}
Let $n$ be a natural number. Are there infinitely many solutions $(a,b,c)$ in $\mathbb{N}$ for:
$$a^2 + b^2 = c^n$$ where say $(a^2 ,b^2, c^n)$ are coprime? Write a function to generate an arbitrary number of such triplets at will. (A generating function/expression). For example, if $n=2$ we may have $(a,b,c) = (k^2-1, 2k, k^2+1)$ etc

\section{Angles around a tetrahedron's corner}
Four distinct points $O,A,B,C$ lie in three-dimensional space and $\angle AOB = c, \angle BOC = a, \angle COA = b$. Prove that
$$1+2\cos a\cos b\cos c \geq \cos^2 a + \cos^2b + \cos^2c$$

\section{exp(x) much?}
Prove that for integers $a,b$ where $a>b \geq 0$
$$\sum_{n=0}^{\infty} \frac {1}{(an+b)!} = \frac {1}{a} \sum_{k=0}^{a-1} \cos \bigg(\sin \bigg( \frac {2 \pi k}{a} \bigg) - \frac {2 \pi b k}{a} \bigg) e ^{\cos \frac {2 \pi k}{a}  }  $$

\section{}
A function $y= f(x)$
is $k$ times differentiable everywhere and satisfies:
$$ \frac {d^k y}{dx^k}= x^{nk} + y$$
Where $n$ and $k$ are natural numbers. Find, in terms of $k$, :

$$\lim_{n\rightarrow \infty} {{f(1)}\over {f(0)}}$$

\section{Descartes theorem (4 mutually touching circles)}
\subsection{}
$a,b,c,d$ are positive real numbers such that
$$\sqrt{\frac{b+c+d}{a}} + \sqrt{\frac{a+c+d}{b}} + \sqrt{\frac{a+b+d}{c}}  = \sqrt{\frac{a+b+c}{d}} $$
Prove that
$$2\left(\frac{1}{a^2} + \frac{1}{b^2} + \frac{1}{c^2} + \frac{1}{d^2}\right) = \left(\frac{1}{a} + \frac{1}{b} + \frac{1}{c} + \frac{1}{d}\right)^2$$

\subsection{}
Three circles (A,B,C) of radius $a,b,c$ are such that each touches the other two externally. If $\Delta$ is the area of the triangle formed with their centres as vertices, if $R$ is the radius of the biggest circle that touches all three, and $r$ is the radius of smallest circle that touches the three, then :
\begin{align*}
    \frac{1}{r} = \frac{1}{a} + \frac{1}{b} + \frac{1}{c} + \frac{2\Delta}{abc}
\end{align*}
If $\frac {1}{\sqrt a} + \frac {1}{\sqrt b} > \frac {1}{\sqrt c}$ then
$$\frac {1}{R} + \frac {1}{a} + \frac {1}{b} + \frac {1}{c} = \frac {2 \Delta}{abc}$$
Other variations will form the same configuration (one of these two), geometrically, but still:
\\\\
If $\frac {1}{\sqrt a} + \frac {1}{\sqrt b} < \frac {1}{\sqrt c}$ $$\frac {1}{R} = \frac {1}{a} + \frac {1}{b} + \frac {1}{c} - \frac {2 \Delta}{abc}$$
If circles A and B touch each other externally and they both touch the bigger circle C internally, then:
$$\frac {1}{R'} = \frac {1}{a} + \frac {1}{b} - \frac {1}{c} \pm  \frac {2 \Delta }{abc}$$
Where the $+/-$ gives the smaller and bigger circle (respectively) that touches all three with radius $R'$.

...etc
\\\\
Descartes theorem states these results in terms of radius of curvatures (which can be positive or negative) which would feel more consistent w.r.t expressions.
\\\\
Also, The incentre (of triangle formed by centres) , the centre of $r$ and the centre of $R$ are collinear. With centre of inversion as the incentre, and inverting radius as inradius, the circles corresponding to $r$ and $R$ invert to each other.

\section{When are cubic roots real?}
$a,b,c$ are real numbers. Prove that:
$$x^3+3ax^2+3bx+c=0$$
Has 3 real roots if and only if
$$(ab-c)^2 \leq 4(ac-b^2)(b-a^2)$$

\section{Blundon's Inequality}
Let $I,O$ be the incentre and circumcentre of a triangle with circumradius $R$ and inradius $r$
Let $s$ be the semi-perimeter.
Prove that:
\begin{align} \bigg | s^2-(2R^2 + 10 Rr - r^2) \bigg | \leq \frac {2 IO^3}{R} \label{blundon}\end{align}
With enough creativity, this can be transformed into a lot of funny looking inequalities. Like:
$$\frac {OH^2}{IO^2} \leq 7 + 2 \frac {IO}{R}$$
which is weaker version of the following equivalent of \eqref{blundon}
$$ \left| \frac {OH^2}{IO^2} + \frac {\sum (a-b)^2}{2 IO^2} - 7 \right| \leq 2 \frac {IO}{R}$$

\section{Sum of $k$-th powers of first $n$ natural numbers}

For $n,k \in \mathbb{N}$ and $k>1, n\geq 1$ let
\begin{align*}
    f_k(n) = \sum_{r=1}^{n} r^k
\end{align*}
Show that
\begin{align*}
    f_k(n) = \begin{cases}
    \frac{n(n+1)}{k+1}\left(n+\frac{1}{2}\right)P_k(n^2 + n) & $k$ \textrm{ is even} \\\\
    \frac{n^2(n+1)^2}{k+1}P_k(n^2 + n) & $k$ \textrm{ is odd} 
    \end{cases}
\end{align*}
where $P_k$ is a $\left\lfloor \frac{k}{2}\right\rfloor-1$ degree polynomial with leading coefficient 1.

For example:
\begin{align*}
    P_2(x) = 1 &,&& P_3(x) = 1 \\
    P_4(x) = x-\frac{1}{3} &,&& P_5(x) = x-\frac{1}{2} \\
    P_6(x) = x^2-x+\frac{1}{3} &,&& P_7(x) = x^2-\frac{4}{3}x+\frac{2}{3}
\end{align*}

\section*{Don't try to prove the following:}
They're easy to verify on a calculator and have little/no purpose otherwise. But they're true. And they're tight. I just wanted to have some content between the problems and solutions so I got creative. These have nothing to do with anything.
\begin{align}
    2(0.9)^{10} > \ln2 &> 0.4^{0.4} \\
    \sqrt{13}+1 &> 2\ln 10 \\
    1.6^{0.2} &< \ln3 \\
    \frac{\pi^2}{10} &> 0.8^{1/17} \\
    5 \ln 2 + 6 \ln 5 -4 \ln 7 - 2 \ln 3 &< \pi < \ln 19 + 20 \ln 10 -23 \ln 7 - \ln 3 \\
    0<31.948^{1/19} - 1.2 &< 10^{-9} \\
    \ln17 + 6\ln 7 &> 7\ln2 + 6\ln 5 \\
    25 &> 12\ln2 + 2\ln3 + 9\ln5 \\
    18\ln3 + 10\ln2 &< 15\ln5+\ln13 \\
    35\ln3 &> 16\ln2+17\ln5 
\end{align}

\section*{Solutions/Hints}
\section*{1}
Someone told me to use Holder's inequality but I didn't know what it was. Anyway, we'll prove:
\begin{align*}
    \left(\frac{x}{t} + \frac{y}{\sqrt{1-t^2}}\right) \geq \left(x^{2/3} + y^{2/3}\right)^{3/2}
\end{align*}
The following inequalities, which are true because of cauchy-schwarz, will be enough:
\begin{align*}
    \left(\frac{x}{t} + \frac{y}{\sqrt{1-t^2}}\right)&\left(t\cdot x^{1/3} + \sqrt{1-t^2} \cdot y^{1/3}\right) \geq \left(x^{2/3} + y^{2/3}\right)^{2} \\
    \left(x^{2/3} + y^{2/3}\right)\left(t^2 + (\sqrt{1-t^2})^2\right) \geq &\left(t\cdot x^{1/3} + \sqrt{1-t^2} \cdot y^{1/3}\right)^2
\end{align*}
We could pre-maturely substitute $x=a\cos^3\phi, y=a\sin^3\phi$ and hope to prove $a\leq 1$ and then these inequalities would look simpler. But whatevs.

\section*{2}
A substitution like $a_n = 2\cos(b_n)$ or $a_n = b_n + (1/b_n)$ with some justification. After that it's ez.

\section*{3}
If $z = x+iy$, let $|z|^2 = x^2+y^2$. Here $x,y$ don't have to be integers. We can write
\begin{align*}
    |(x+iy)^n|^2 &= (|x+iy|^2)^n \\
    |p_n(x,y) + i q_n(x,y)|^2 &= (x^2+y^2)^n \\
    \left(p_n(x,y)\right)^2 + \left(q_n(x,y)\right)^2 &= (x^2+y^2)^n
\end{align*}
Here $p_n$ and $q_n$ are polynomials one will get from the expansion of $(x+iy)^n$. It can be written similar to how a binomial expansion is written but who has the time. For this $x,y$ don't have to be integers, but by choosing integers $x,y$ we can generate $(a,b,c) = (p_n(x,y), q_n(x,y), x^2+y^2)$. For instance, if $n = 3$, we can have $(a,b,c) = (k^3 - 3k, 3k^2 - 1, k^2+1)$

\section*{4}
LHS$-$RHS = $\bigg(\overrightarrow{OA} \cdot (\overrightarrow{OB} \times \overrightarrow{OC})\bigg)^2$. Write in terms of the determinant for clarity.

\section*{5}
Say the $a$-th roots of unity are $\omega_k$ for $k\in \{0,1,\dots,a-1\}$. Then look at the expansion of $\exp(\omega_k)$ (or $e^{\omega_k}$) for all these $k$. What happens when we add them up? So what do we do when $b\neq0$?

\section*{6}
The answer, to the best of my knowledge, is the expression from 5 with $a=k,b=0$. To get that, suppose you find one such function $y_0 = f_0(x)$ satisfying the given differential equation. Then $y-y_0$ is the $k$-th derivative of itself and we know how to solve for that. Now to guess such a $y_0$, think about $k=1$. Now how will you do it for $k>1$?

\section*{7}
7.1 is just the exact algebraic problem posed by 7.2 where $d=r$. One method to derive these relations (in contrast to a proof) in 7.2 if we don't know them beforehand is by inversion (geometry). Circles invert to circles but sometimes to straight lines. If they touch before inversion they touch after inversion. I don't want to draw any figures, so go figure (hah) the centre and radius of inversion that might help.

\section*{8}
Following cardano's method to find cubic roots, the substitution $x = y-a$ and $y = z + (a^2 - b)/z$ will give a quadratic in $z^3$. The existence of real roots of this quadratic is related to the existence of 3 real roots of the original cubic. What happens for the case of equal roots in the quadratic?

\section*{9}
If the sides of the triangle are $a,b,c$ then
\begin{align*}
    a+b+c &= 2s\\
    ab+bc+ca &= r^2 + s^2 + 4Rr \\
    abc &= 4Rrs
\end{align*}
Let $p(x) = (x-a)(x-b)(x-c)$. If we impose the constraint (on the coefficients of $p(x)$) that it has all 3 roots real (like in 1.8), then we get an inequality in $R,r,s$ which with enough algebra (and motivation) can be converted to Blundon's inequality. Also, we only need some three properties of the triangle that can be expressed in terms of $R,r,s$ and those quantities need not be the sides. For instance, take the radii of the three ex-circles of a triangle as $r_1,r_2,r_3$. We can write
\begin{align*}
    r_1 + r_2 + r_3 &= r+4R\\
    r_1r_2 + r_2r_3 + r_3r_1 &= s^2\\
    r_1r_2r_3 &= rs^2
\end{align*}
Imposing three-real-roots-constraint on coefficients of $p(x) = (x-r_1)(x-r_2)(x-r_3)$ instead will also lead to the same inequality (eventually).

\section*{10}
I did not come up with this, but to make an inductive argument (this also doubles up as a recursive algorithm to find $P_k$) consider this:
\\\\
$r^p(r+1)^p - r^p(r-1)^p$ is a polynomial in $r$ with only odd powers starting at $2p-1$. Sum up this expression as $r$ ranges from $1$ to $n$, in two different ways. This will help to find $P_k$ for $k = 2p-1$.
\\\\
Similarly, $r^p(r+1)^p(2r+1) - r^p(r-1)^p(2r-1)$ is a polynomial in $r$ with only even powers starting at $2p$. Sum up this expression as $r$ ranges from $1$ to $n$, in two different ways. This will help to find $P_k$ for $k = 2p$.
\\\\
As far as recursions go, this one sucks because to find $P_k$ we need roughly $k/4$ instance of $P_i$ for different $i<k$. But it's better than to require all $\sim k-1$ instances in the traditional approach where the polynomial in question is $(r+1)^p-r^p$

\end{document}